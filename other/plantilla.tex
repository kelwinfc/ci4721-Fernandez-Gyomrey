% file thesis.tex
% Archivo thesis.tex
% Documento maestro que incluye todos los paquetes necesarios para el documento
% principal.

% Documento obtenido por un sinfin de iteraciones de administradores del LDC
% Estructura actual hecha por:
% Jairo Lopez <jairo@ldc.usb.ve>
% Actualizado ligeramente por:
% Alexander Tough 

\documentclass[oneside,12pt,letterpaper]{article}
\tolerance=1000  
\hbadness=10000  
\raggedbottom

% Para escribir algoritmos
\usepackage{listings}
\usepackage{algpseudocode}
\usepackage{algorithmicx}
\usepackage{algorithm}

% Paquetes para manejar graficos
\usepackage{epsf}
\usepackage[pdftex]{graphicx}
\usepackage{epsfig}
% Simbolos matematicos
\usepackage{latexsym,amssymb}
% Paquetes para presentar una tesis decente.
\usepackage{setspace,cite} % Doble espacio para texto, espacio singular para
                           % los caption y pie de pagina
\usepackage[table]{xcolor}
\usepackage{tikz}
\usetikzlibrary{arrows,shapes}
\usepackage{verbatim}

\usepackage{comment}

% Paquetes no utilizados para citas
%\usepackage{mcite} 
%\usepackage{draft} 

\usepackage{wrapfig}
\usepackage{alltt}

% Acentos 
\usepackage[spanish,activeacute,es-noquoting]{babel}

\usepackage[spanish]{translator}
\usepackage[utf8]{inputenc}
\usepackage{color, xcolor, colortbl}
\usepackage{multirow}
\usepackage{subfig}
\usepackage[OT1]{fontenc}
\usepackage{tocbibind}
\usepackage{anysize}
\usepackage{listings} 

% Para poder tener texto asiatico
%\usepackage{CJK}

% Opciones para los glosarios
\usepackage[style=altlist,toc,numberline,acronym]{glossaries}
\usepackage{url}
\usepackage{amsthm}
\usepackage{amsmath}
\usepackage{fancyhdr} % Necesario para los encabezados
\usepackage{fancyvrb}
\usepackage{makeidx} % En caso de necesitar indices.
\makeindex  % Necesitado para los indices

% Definiciones para definicions, teoremas y lemas
\theoremstyle{definition} \newtheorem{definicion}{Definici\'{o}n}
\theoremstyle{plain} \newtheorem{teorema}{Teorema}
\theoremstyle{plain} \newtheorem{lema}{Lema}

% Para la creacion de los pdfs
\usepackage{hyperref}

% Para resolver el lio del Unicode para la informacion de los PDFs
% En pdftitle coloca el nombre de su proyecto de grado/pasantia.
% En pdfauthor coloca su nombre.
\hypersetup{
    pdftitle = {Lenguajes de Programaci'on III},
    pdfauthor={Kelwin Fernández, Andras Gyomrey},
    colorlinks,
    citecolor=black,
    filecolor=black,
    linkcolor=black,
    urlcolor=black,
    backref,
    pdftex
}

\definecolor{brown}{rgb}{0.7,0.2,0}
\definecolor{darkgreen}{rgb}{0,0.6,0.1}
\definecolor{darkgrey}{rgb}{0.4,0.4,0.4}
\definecolor{lightgrey}{rgb}{0.95,0.95,0.95}

\usepackage{listings}
\lstnewenvironment{code}{\lstset{basicstyle=\small}}{}

\lstset{
   frame=single,
   framerule=1pt,
   showstringspaces=false,
   basicstyle=\footnotesize\ttfamily,
   keywordstyle=\textbf,
   backgroundcolor=\color{lightgrey}
}

% Pone los nombres y las opciones para mostrar los codigos fuentes
\lstset{language=C, breaklines=true, frame=single, showstringspaces=false,
        showtabs=false, numbers=left, keywordstyle=\color{black},
        basicstyle=\footnotesize, captionpos=b }
\renewcommand{\lstlistingname}{C\'{o}digo fuente}
\renewcommand{\lstlistlistingname}{\'{I}ndice de c\'{o}digos fuentes}

\newcommand{\todo}{ TODO: }

% Dimensiones de la pagina
\setlength{\headheight}{15pt}
\marginsize{3cm}{2cm}{2cm}{2cm}

%%%%%%%%%%%%%%%%%%%%%%%%%%%%%%%%%%%%%%%%%%%%%%%%%%%%%%%%%%%%%%%%%%%%%%%%%%%
%%%%%%%%%%%%%%%%      end of preamble and start of document     %%%%%%%%%%%
%%%%%%%%%%%%%%%%%%%%%%%%%%%%%%%%%%%%%%%%%%%%%%%%%%%%%%%%%%%%%%%%%%%%%%%%%%%
\begin{document}


\title{Documento Autogenerado de Estado de Compilaci'on}

\author{BLA Compiler}

\maketitle

\pagebreak

\section{Descripci'on de Tipos}
\begin{verbatim}
 TYPES:
  * _undefined [size 0][alignment 1]
  * _invalid [size 0][alignment 1]
  * none [size 0][alignment 1]
  * int [size 4][alignment 4]
  * float [size 4][alignment 4]
  * char [size 1][alignment 1]
  * boolean [size 1][alignment 1]
  * string [size 4][alignment 4]
\end{verbatim}

\section{Constantes String}
\begin{verbatim}
- "hola"
\end{verbatim}

\section{'Arbol Sint'actico Abstracto}

\begin{verbatim}
  5: Declaration: c (int) [offset 0] initialized with
  5:     Constant int: 30
  6: Declaration: d (int) [offset 4] initialized with
  6:     Constant int: 30
 12: Declaration of Function fibRec -> int
 12:     (
 12:     int(n)
 12:     )
 13:     {
 13:         If
                 (
 13:             == [boolean]
 13:                 Identifier: n [int:12:20]
 13:                 Constant int: 0
                 )
 14:             {
 14:                 Return [int]
 14:                     Constant int: 0
 14:             }
 15:         Else If
                 (
 15:             == [boolean]
 15:                 Identifier: n [int:12:20]
 15:                 Constant int: 1
                 )
 16:             {
 16:                 Return [int]
 16:                     Constant int: 1
 16:             }
 18:         Return [int]
 18:             + [int]
 18:                 fibRec [int:12:9]
                     (
 18:                     - [int]
 18:                         Identifier: n [int:12:20]
 18:                         Constant int: 1
                     )
 18:                 fibRec [int:12:9]
                     (
 18:                     - [int]
 18:                         Identifier: n [int:12:20]
 18:                         Constant int: 2
                     )
 18:     }
 21: Declaration: b (int) [offset 8] initialized with
 21:     Constant int: 42
 22: Declaration: e (int) [offset 12] initialized with
 22:     Constant int: 42
 48: Declaration: a (int) [offset 16] initialized with
 48:     Constant int: 42
 50: Declaration of Function main -> none
 50:     (
 50:     )
 51:     {
 51:         Declaration: a (int) [offset 0] unitialized
 52:         Read a [int:51:9]
 53:         Declaration: fR (int) [offset 4] initialized with
 53:             fibRec [int:12:9]
                 (
 53:                 Identifier: a [int:51:9]
                 )
 55:         Declaration: salida (string) [offset 8] initialized with
 55:             Constant string: hola
 55:     }
\end{verbatim}

\end{document}
